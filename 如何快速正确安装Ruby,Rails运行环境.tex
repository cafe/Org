% Created 2012-07-08 日 17:25
\documentclass[11pt]{article}
\usepackage[utf8]{inputenc}
\usepackage[T1]{fontenc}
\usepackage{fixltx2e}
\usepackage{graphicx}
\usepackage{longtable}
\usepackage{float}
\usepackage{wrapfig}
\usepackage{soul}
\usepackage{textcomp}
\usepackage{marvosym}
\usepackage{wasysym}
\usepackage{latexsym}
\usepackage{amssymb}
\usepackage{hyperref}
\tolerance=1000
\providecommand{\alert}[1]{\textbf{#1}}

\title{如何正确的安装 Ruby, Rails运行环境}
\author{hdc}
\date{\today}
\hypersetup{
  pdfkeywords={},
  pdfsubject={},
  pdfcreator={Emacs Org-mode version 7.8.11}}

\begin{document}

\maketitle

\setcounter{tocdepth}{3}
\tableofcontents
\vspace*{1cm}
对于新入门的开发者,如何安装 Ruby,Ruby Gems 和 Railes 的运行环境可能会是个问题,本页主要介绍如何用一条靠谱的路子快速安装 Ruby 开发环境。
此安装方法同样适用于产品环境!


\section{\textbf{系统要求}}
\label{sec-1}

首先确定操作系统环境,不建议在windows上面搞,所以你需要用:
\begin{itemize}
\item Mac OS X
\item 任意 Linux 发行版本(Ubuntu,CentOS,Redhat,ArchLinux\ldots{})
\end{itemize}
\textbf{以下代码区域,带有 \$ 打头的表示需要在控制台(终端)下面执行(不包括 \$ 符号)}
\subsection{步骤0 - 安装系统需要的包}
\label{sec-1-1}

\textbf{Mac 请安装 Xcode 开发工具,安将帮你安装好 Unix 环境需要的开发包}
\textbf{Ubuntu 安装,开发包}

\begin{verbatim}
$ sudo apt-get install wget vim build-essential openssl libreadline6 libreadline6-dev curl git-core zlib1g zlib1g-dev libssl-dev libyaml-dev libxml2-dev libxslt-dev autoconf automake libtool imagemagick libpcre3-dev
\end{verbatim}
\subsection{步骤1-安装 RVM}
\label{sec-1-2}

RVM 是干什么的这里就不解释了,后面你将会慢慢搞明白。

\begin{verbatim}
$ curl -L get.rvm.io | bash -s stable
\end{verbatim}
等待一段时间后就可以成功安装好 RVM。

\begin{verbatim}
$ source ~/.bash_profile
\end{verbatim}
测试是否安装正确

\begin{verbatim}
$ rvm -v
rvm 1.14.3 (stable) by Wayne E. Seguin <wayneeseguin@gmail.com>, Michal Papis <mpapis@gmail.com> [https://rvm.io/]
\end{verbatim}
\subsection{步骤2-用 RVM 安装 Ruby 环境}
\label{sec-1-3}


\begin{verbatim}
$ rvm pkg install readline
$ rvm install 1.9.2 --with-readline-dir=$rvm_path/usr
\end{verbatim}
或者可以安装1.8.7版本,也可以是1.9.3,只要将后面的版本号更换一下就可以了
同样继续等待漫长的下载,编译过程,完成以后,Ruby,Ruby Gems 就安装好了。
\subsection{步骤3-设置 Ruby 版本}
\label{sec-1-4}

RVM 装好以后,需要执行下面的命令将指定版本的 Ruby 设置为系统默认版本

\begin{verbatim}
$ rvm 1.9.3 --default
\end{verbatim}
同样,也可以用其他版本号,前提是你有用 rvm install 安装过那个版本
这个时候你可以测试是否正确

\begin{verbatim}
$ ruby -v
ruby 1.9.3p194 (2012-04-20 revision 35410) [x86_64-linux]

$ gem -v
1.8.24

$ gem source -r http://rubygems.org/
$ gem source -a http://ruby.taobao.org
\end{verbatim}
\subsection{步骤4-安装 Rails 环境}
\label{sec-1-5}

上面3步过后,Ruby环境就安装好了,接下来安装 Rails

\begin{verbatim}
$ gem install bundler rails
\end{verbatim}
然后测试安装是否正确

\begin{verbatim}
$ bundle -v
Bundler version 1.1.4

$ rails
\end{verbatim}

\end{document}
